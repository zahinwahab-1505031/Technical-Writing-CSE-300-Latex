\documentclass{article}
\usepackage[utf8]{inputenc}
\usepackage{color}
\usepackage{graphicx}
\title{online}
\author{zahinwahab }
\date{April 2018}

\begin{document}

\maketitle

\section{Simpson's 3/8 rule}
Simpson's 3/8 rule is another method for numerical integration proposed by Thomas Simpson. It is based upon a cubic interpolation rather than a quadratic interpolation.It is shown in figure \ref{fig:fig1}.Simpson's 3/8 rule is as follows:
\begin{equation}
\int_{a}^{b} f(x) dx \approx \frac{3h}{8}[f(a)+3f(\frac{2a+b}{3})+3f(\frac{a+2b}{3})+f(b)]=\frac{(b-a)}{8}[f(a)+3f(\frac{2a+b}{3})+3f(\frac{a+2b}{3})+f(b)]
\end{equation}

where $b − a = 3h$. The error of this method is:
$

-\frac{(b-a)^5}{6480}f^(4)(\xi)
$

where $ \xi$  is some number between $ a$ and   $ b$. Thus, the 3/8 rule is about twice as accurate as the standard method, but it uses one more function value. A composite 3/8 rule also exists, similarly as above.[6]

A further generalization of this concept for interpolation with arbitrary-degree polynomials are the \textcolor{blue}{Newton–Cotes formulas}.
\begin{figure}[h]
\label{fig:fig1}
  \centering
   \includegraphics[width=0.4\textwidth]{figures/Simpsons_method.png}
    \caption{Simpson's 3/8 rule}
    \label{fig:my_label}
\end{figure}
\end{document}
