\documentclass{article}
\usepackage[utf8]{inputenc}
\usepackage{subfiles}
\usepackage{color}
\usepackage{tikz}
\usepackage{textgreek}
\usetikzlibrary{arrows.meta}
\usepackage{amsmath}
\usepackage{amssymb}
\usepackage{mathtools}
\usepackage{setspace}
\newtheorem{theorem}{Theorem}[section]
\definecolor{background}{rgb}{0.84, 0.92, 0.95}
\definecolor{myCyan}{rgb}{0.11, 0.71, 0.86}
\definecolor{myPurple}{rgb}{0.64, 0.46, 0.66}
\definecolor{myOrange}{rgb}{0.96, 0.66, 0.59}
\definecolor{textColor}{rgb}{0.31,0.20,0.53}
\definecolor{lineColor}{rgb}{0.68,0.45,0.45}
\definecolor{myRed}{rgb}{0.91,0.29,0.31}
\definecolor{circleBorder}{rgb}{0.09,0.17,0.39}
\definecolor{diameterColor}{rgb}{0.22,0.29,0.42}
\definecolor{myLightBlue}{rgb}{0.76,0.84,0.95}
\title{Pythagorean Theorem}
\author{1505031 }
\date{\today}


\begin{document}

\maketitle

\section{Introduction}
In this document, we present the very famous theorem in mathematics: \textit{Pythagorean theorem}, which is stated as follows.

\begin{theorem}[Pythagorean theorem]
The square of the hypotenuse (the side opposite the right angle) is equal to the sum of the squares of the other two sides.
\end{theorem}

Numerous mathematicians proposed various proofs to the theorem. The theorem was long known even before the time of Pythagoras. Pythagoras was the first to provide with a sound proof. The proof that Pythagoras gave was by \textit{rearrangement}. Even the great Albert Einstein also proved the theorem without rearrangement, rather by using dissection. Figure 1 shows the visual representation of the theorem.
\subfile{figures/1505031_fig1.tex}
\subfile{figures/1505031_fig2.tex}
\section{Trigonometric Forms}
Lots of other forms of the same theorem exist. The most useful, perhaps, are
expressed in trigonometric terms, as follows:

\begin{equation}
\sin^2\theta + \cos^2\theta = 1
\label{eqn:1}
\end{equation}
\begin{equation}
\sec^2\theta - \tan^2\theta = 1
\label{eqn:2}
\end{equation}
\begin{equation}
\mathrm{cosec^2}\theta - \cot^2\theta = 1
\label{eqn:3}
\end{equation}

\subsection{Representing the First}
Taking 1, we can show them as shown in Figure2. When we take a point at
unit distance from the origin, the $y$ and $x$ co-ordinates become $\sin\theta$ and $\cos\theta$ respectively. Therefore, sum of the squares of the two becomes equal to the
square of the unit distance, which of course, is 1. %fig 2 ke label

\end{document}