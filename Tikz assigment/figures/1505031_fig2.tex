\begin{figure}[h]
    \centering
    \begin{tikzpicture}[scale=0.3]
  
    
        \fill[opacity=0] (-13,-6.6) rectangle (6.7,10);
       

\draw[thick,color = myLightBlue] (-5.2,4.7) -- (1.4,4.7);
\draw[thick,color = myLightBlue] (1.3,1.65) -- (1.3,4.7);
\draw[thick,color = myLightBlue] (-2,1.6) arc (0:27:3.18);
%circle ar majher duita
\draw[thick,color=circleBorder] (-13.55,1.65) -- (3.15,1.65);
\draw[thick,color=circleBorder] (-5.2,10) -- (-5.2,-6.7);
\draw[ultra thick,color = circleBorder] (-5.2,1.65) circle [radius=7.13];
%red color er line ta
\draw[thick,color =myRed] (-5.2,1.65) -- (1.3,4.7);
%dimension line
\draw[thick]  (1.3,4.7) -- (3.15,4.7);
\draw[thick]  (1.3,4.7) -- (1.3,6.55);
%arrow
\draw[circleBorder, arrows={-Triangle[angle=90:5pt,circleBorder,fill=circleBorder]}]  (2.5,1.65) -- (2.5,4.7);
\draw[circleBorder, arrows={-Triangle[angle=90:5pt,circleBorder,fill=circleBorder]}]   (2.5,4.7) -- (2.5,1.65);


\draw[circleBorder, arrows={-Triangle[angle=90:5pt,circleBorder,fill=circleBorder]}]  (-5.2,5.9) -- (1.4,5.9);
\draw[circleBorder, arrows={-Triangle[angle=90:5pt,circleBorder,fill=circleBorder]}]   (1.4,5.9) -- (-5.2,5.9);

%text gula
 \node[above] at (-1.5,1.5) {\textbf{\large \straighttheta}};
 \node[right] at (2.6,3.175) {\textbf{\large$\sin$\straighttheta}};
  \node[above] at (-2.3,5.9){\textbf{\large$\cos$\straighttheta}};
  \node[above,color=myRed,rotate=27] at ( (-2,3) {\textbf{\small{ \textit{r=1}}}};
\end{tikzpicture}
     \caption{ Alternate representation of Pythagorean theorem.}
\end{figure} 